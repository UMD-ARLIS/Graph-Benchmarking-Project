\documentclass[letterpaper, 10pt]{article}

% % % % % % % % % % % % % %
% Preamble
% % % % % % % % % % % % % %
 
%Get all the formatting details from preamble.tex
% % % % % % % % % % % % % %
% Package Imports
% % % % % % % % % % % % % %
%\usepackage[paperheight=28cm, paperwidth=22cm, includehead,
%nomarginpar, textwidth=18cm, headheight=20mm,footheight=20mm]{geometry}
\usepackage[letterpaper,textwidth=180mm,top=40mm,textheight=210mm]{geometry}
%\geometry{showframe=true}
\usepackage{graphicx}   % Required for inserting images
\usepackage{fancyhdr}   % Required for headers and footers
\usepackage{titling}    % Required to reformat the title
\usepackage{multicol}   % Required to use multiple columns in the doc
\usepackage[dvipsnames]{xcolor}     % Required to change the color of section headings
\usepackage{sectsty}    % Required to change the color of section headings
\usepackage{lipsum}     % Lorem Ipsum generator, can be removed. 

% % % % % % % % % % % % % %
% Macros and Custom Commands
% % % % % % % % % % % % % %

%Update the title header. 
\renewcommand\maketitlehooka{%
  \setlength\parindent{0pt}%
  %\begin{minipage}{\textwidth}
    \begin{minipage}[b][.10\textheight][t]{\textwidth}
        \centering \small{\textbf{SECURITY//LABELS//HERE}}\\
    \begin{minipage}{.50\textwidth}
      \includegraphics[width=\textwidth]{arlis}
    \end{minipage}%
    \begin{minipage}{.50\textwidth}
      \raggedleft
      \textbf{Research for Intelligence \& Security Challenges }\par
      \textbf{Summer 2023 Internship Program}\par
      \textbf{Midprogram Status Report }\par
    \end{minipage}%
  \end{minipage}
  \par
  }

%Make header and footer hlines invisible
\renewcommand{\headrulewidth}{0pt} 
\renewcommand{\footrulewidth}{0pt}

%Left-Align the Abstract
\renewcommand*\abstractname{\flushleft{\textbf{Abstract} }}

% % % % % % % % % % % % % %
% Document Config
% % % % % % % % % % % % % %

%Tell LateX where to find images
\graphicspath{ {./Figures/} }

%Title Format
\setlength{\droptitle}{-100pt}
\pretitle{\begin{flushleft}\LARGE}
\posttitle{\par\end{flushleft}\vspace{-1em}}
\preauthor{\begin{flushleft}\large}
\postauthor{\par\end{flushleft}\vspace{-1.5em}}
\predate{\begin{flushleft}\large}
\postdate{\par\end{flushleft}\vspace{-2em}}

%Make the section and subsection headings red
\sectionfont{\color{Mahogany}}  % sets colour of sections
\subsectionfont{\color{Mahogany}}



\begin{document}
\pagestyle{empty}

%Make the title block
\title{\color{arlisRed}{
    \LARGE{Unlocking the Power of Hierarchical Identify Verify Exploit (HIVE): Revolutionizing Data Analysis and Efficiency }\\ 
    \large{Team MN}}}
\author{                                                            %Add as many authors as you like on new lines   
    \color{arlisRed}{
        Nandini Ramachandran, University of Maryland, nandinir@umd.edu}\\
        Morein Ibrahim, University of Maryland, morein04@terpmail.umd.edu \\
        Kent O'Sullivan, University of Maryland, osullik@umd.edu\\
    \small{\color{black}{                                           %Add as many POCs / mentors as you like on new lines
        \textbf{Sponsoring Agency:} DARPA\\
        \textbf{RISC Faculty Mentor 1:} William Regli, University of Maryland, regli@umd.edu \\
        \textbf{RISC Faculty Mentor 2:} Cliston Cole, Morgan State University, Cliston.cole@morgan.edu \\ 
        \textbf{Project Github:} \href{https://github.com/osullik/summer2023}{https://github.com/osullik/summer2023} [Private - Email a Team Member for Access]
    }} 
}

% % % % % % % % % % % % % %
% Headers and footers
% % % % % % % % % % % % % %
\pagestyle{fancy}
\fancyhead{}        %Flush the header and footer
\fancyfoot{}
%Header
\fancyhead[C]{\small{\textbf{UNCLASSIFIED}}\vspace{55pt}}
%Footer
\fancyfoot[C]{\thepage \\ Approved for public release: distribution is unlimited.\\ © 2023 UMD/ARLIS. All Rights Reserved. Proprietary Information. \textbf{\\UNCLASSIFIED} }


% % % % % % % % % % % % % %
% Document Content
% % % % % % % % % % % % % %

\maketitle

\abstractname{}
\par{\setlength{\parindent}{20pt}
Graph analytics has emerged as a critical tool for intelligence and security applications, enabling the analysis of complex relationships and patterns in data. 
The DARPA HIVE project aims to develop an advanced graph analytics processor that surpasses the current state of the art in graph processing speed and power consumption. 
Here, we present the beginnings of a framework that evaluates the performance of graph processing algorithms relevant to the intelligence and security community such as community detection, subgraph matching, and knowledge graphs on CPU, GPU, and PiUMA architectures. 
The report emphasizes the importance of graph analysis in intelligence and security domains, outlines the methodology for evaluating performance using different algorithms and hardware configurations, discusses reference algorithm implementations, and highlights the study's progress. 
The report aims to provide valuable insights into PiUMA's capabilities and contribute to advancements in graph processing technologies for intelligence and security applications.}

\begin{multicols}{2}    
    \section{Project Goals}\label{section:goals}
    Team Minnesota's project sits within the Hierarchical Identify Verify Exploit (HIVE) program overseen by the Defense Advanced Research Project Agency (DARPA)\footnote{\href{https://www.darpa.mil/program/hierarchical-identify-verify-exploit}{https://www.darpa.mil/program/hierarchical-identify-verify-exploit}}. The ARLIS Statement of Work (SOW) identifies several goals for the project, discussed below. 
        \subsection{Overarching Project Goal}\label{section:projectGoal}
        The HIVE program seeks to develop an advanced graph processor capable of efficiently processing streaming graphs 1000x faster while reducing power consumption. 
        Team Minnesota's role within the program is to develop a benchmarking suite to measure the progress of Intel's Programmable Integrated Unified Memory Architecture (PIUMA) and any other following efforts towards the performance goals. 
        The benchmark suite must be specific to evaluating performance on problems specific to the intelligence and security domain.
        \subsection{Sub Goals}\label{section:subGoals}
        \begin{enumerate}
        \item Collect datasets that represent intelligence and security problems typically approached with graph analysis. 
        \item Source or develop suitable cross-architecture reference implementations of graph analysis algorithms commonly used by the intelligence and security community. 
        \item Construct an experimental framework that can report performance in terms of execution time, memory usage, and power consumption.
        \item Benchmark PiUMA, a new architecture for high-performance graph processing against the provided projections 
        \end{enumerate}

        \subsection{Relevance to the Intelligence and Security Community} \label{section:relevance}
        
        \par{Many intelligence and security problems are fundamentally social problems. 
        Whether the goal is to determine the axis of advance for an enemy tank battalion, tracing transnational assassins\footnote{\href{https://www.bellingcat.com/resources/2020/12/14/navalny-fsb-methodology/}{https://www.bellingcat.com/resources/2020/12/14/navalny-fsb-methodology/}} identifying foreign intelligence officers\footnote{\href{https://www.wired.com/2007/07/in-italy-cia-agents-are-undone-by-their-cell-phones/}{https://www.wired.com/2007/07/in-italy-cia-agents-are-undone-by-their-cell-phones/}}, analyzing disinformation operations\footnote{\href{https://www.bellingcat.com/news/2019/09/03/twitter-analysis-identifying-a-pro-indonesian-propaganda-bot-network/}{https://www.bellingcat.com/news/2019/09/03/twitter-analysis-identifying-a-pro-indonesian-propaganda-bot-network/}} unmasking members of the GRU\footnote{\href{https://www.bellingcat.com/news/uk-and-europe/2020/10/22/russian-vehicle-registration-leak-reveals-additional-gru-hackers/}{https://www.bellingcat.com/news/uk-and-europe/2020/10/22/russian-vehicle-registration-leak-reveals-additional-gru-hackers/}} or trying to prevent human trafficking \cite{Szekely2015}. In each case, an analyst is trying to understand human behaviour, and their interactions with other humans and their environment. We can model these human problems as a collection of social interactions between humans. 
        Using a graph data structure is a powerful method of modeling human social behavior.}
        
        \par{Formally, a Graph $G$ is a collection of Vertices (or nodes) $V$ and edges $E$ such that $G=\{V, E\}$.
        Consider the $PhoneRecords$ graph in figure \ref{fig:callRecords}. $PhoneRecords$ is made up of Vertices $PhoneNumbers$ and Edges $PhoneCalls$ that link $PhoneNumbers$ to each other.
        So, $PhoneRecords$ is a graph containing the set of all Phone Numbers and the calls between those numbers.}

        \begin{figure*}
            \centering
            \includegraphics[width=\columnwidth]{Papers/ARLIS_MPR_MN/Figures/PhoneRecords.png}
            \caption{A Notional $CallRecords$ graph has phone numbers as the $Vertexes$ (or $nodes$) as phone numbers and the calls between those numbers as $Edges$. Suppose the analyst knows the phone number of a person of interest (in red). In that case, they can use graph analytics to address intelligence requirements about the community the POI is involved with, where call patterns correlate with events, or what possible multi-hop information flows could be.}
            \label{fig:callRecords}
        \end{figure*}
        
        %Because of their powerful ability to model and query social interactions of intelligence interest, graphs are used extensively across the intelligence community to store and analyze data. 
        \par{Graph analysis is implemented through algorithms and techniques designed to process and analyze graph data, like our $PhoneRecords$ graph in Figure \ref{fig:callRecords} at a much larger scale. 
        These algorithms traverse, explore, and extract insights from the interconnected nodes and edges of the graph. By identifying hidden patterns, facilitating decision-making, and enhancing situational awareness, graph analysis contributes to a nation's overall security and defense efforts.} 
        
        \par{The problem motivating the HIVE program is that when these graphs get very large (say, if $PhoneCalls$ has all phone calls made by all numbers in the USA for the last 50 years), they become very slow to search, and many problems quickly become intractable.
        Just as the advent of the GPU has breathed life into the neural network algorithms developed in the 80s and 90s by increasing the amount of available compute \cite{Dally2021}, HIVE expects that specialized hardware optimized for Graph Processing may make some of these 'intractable' problems solvable, and expand the toolkit available to the I\&S community.} 
        
    \section{Background and Related Work}\label{section:background}
        \par{Towards determining whether the PiUMA system is a step towards the HIVE goal, we consider our research in the context of prior work in graph workload benchmarking. 
        We summarize our preliminary review in terms of the available Datasets, Reference Algorithm implementations in Table \ref{table:graphAlgorithms}, supported metrics in Table \ref{table:graphMetrics}, and available System Architectures in Table \ref{table:graphArchitectures}.}

        \subsection{Datasets}\label{section:datasets}
        \par{The broad consensus in the literature is that 'real' datasets give benchmarks credibility. However, only the GAP Benchmark provides 'real' data \cite{Beamer2017}. 
        Synthetic data dominates, with the 2004 Recursive Matrix algorithm \cite{Chakrabarti2004} and its successors like the Kroenecker graph generator \cite{Leskovec2010} core among them. 
        More recent approaches, including the Social Dataset Generator \cite{Angles2013} and Datagen \cite{Capota2015}, attempt to extend the graph generators to support streaming data and structures more representative of 'real' social networks.
        There appears to be no standard statistical description of graph datasets. 
        Using vertex and edge count is ubiquitous but does not communicate structure. 
        %R-MAT derivatives like the Graph-500 dataset \cite{Murphy2010} tend to favor the $a+b+c+d=1.0$} parameter set for synthetic data, describing the probability distribution of vertices across an adjacency matrix.
        More useful metrics are introduced by the team who developed the graphalytics benchmark, including cluster coefficient, assortativity, and distribution fit \cite{Capota2015}.
        Overall, our project must aim to select standard, real-world datasets and find suitable metrics to describe the structure of our graphs so that we can fairly compare performance between different datasets.
        

        \scriptsize
        \begin{table*}[h]
        \centering
          \begin{tabular}{ |c|c|c|c|c|c|c|c|c|c|c|}
            \hline
            {Benchmark Suite} & \multicolumn{10}{|c|}{Algorithm}\\
            \hline
                                                      & STATS & BFS & SSSP & PR & CC & BC & TC & CD & SGM & KGA \\
            \hline
             Graph500 [2010]\cite{Murphy2010}         &       & X   &      &    &    &    &    &    &     &      \\
             Social Benchmark [2013]\cite{Angles2013} &   X   &     &      &    &    &    &    &    &     &  X   \\
             Graphalytics [2015]\cite{Capota2015}     &   X   & X   &      &    & X  &    &    &  X &     & ?    \\
             GAP [2017]\cite{Beamer2017}              &       & X   & X    & X  & X  & X  & X  &    &     &      \\
            \hline
          \end{tabular}
          \caption{Summary of Graph Benchmark Algorithms.\\ STAT = Statistics, BFS = Breadth First Search, SSSP = Single Source Shortest Path, PR = PageRank, CC = Connected Components, BC = Betweenness Centrality, TC = Triangle Counting, CD = Community Detection, SGM = Sub Graph Matching, KGA = Knowledge Graph Analytics}
          \label{table:graphAlgorithms}
        \end{table*}

        \normalsize

        \subsection{Supported Metrics}\label{section:metrics}
        %To fairly compare each implementation on each architecture, we need to define standard metrics to evaluate performance. 
        The Statement of Work and original PiUMA Paper identify execution speed as \textit{Traversed Edges Per Second (TEPS)} and power consumption as \textit{TEPS per Watt (TEPS/W)} as the metrics to optimize \cite{Aananthakrishnan2020}. 
        TEPS is not widely reported in existing benchmarks, with only Graphalytics referring to it, describing their calculation as $\frac{Total Execution Time}{Number of Edges in Graph}$ \cite{Capota2015}, which will not account for multiple traversals of the same edges, and so it unlikely to give accurate measurements for iterative algorithms like Louvain.
        The most common metric is \textit{Execution Time} per query, with all surveyed approaches reporting it. 
        Measurement of load time and the Objects per Second Load rate is only tracked by the Social Network Benchmark \cite{Angles2013}, with it explicitly scoped out by most approaches. 
        %As HIVE focuses on streaming graph problems, there is a valid argument that ETL time should not be explicitly measured. 
        %However, understanding the processor and memory implications of insertions, deletions, and other updates will be very relevant, with the Social Network benchmark identifying the need as far back as 2013 \cite{Angles2013}.
        Given the repeated assertions that graph applications are bound by memory latency and particularly hurt by a lack of spatial and temporal locality resulting from their sparse structure \cite{Mutlu2023, Ren2010, Blondel2008, Capota2015, Beamer2017}, it is surprising that there is no straightforward, common approach to measuring spatial and temporal locality of graphs in memory.
        %That may be because of the difficulty in measuring locality scores. For example, in 2005, Weinberg et al. argued that reducing locality to a simple scalar score is overly reductive before immediately introducing their own scalar score for locality.
        Recent studies indicate that 62\% of all power usage is attributed to moving data to and from memory \cite{Mutlu2023} and that 95\% of systems use less than 31\% of their memory bandwidth \cite{Kanev2015} because of latency issues fetching data from memory there is a clear need to characterize the spatial and temporal locality of graphs as part of a benchmark. 
        %Spatial and temporal locality has knock-on effects for execution time and power consumption, the metrics we care about.
        To generate meaningful metrics to compare each implementation, we need to identify a standard to count the traversed edges per second, measure total execution time, measure spatial locality, measure temporal locality, and measure power consumption given each of these other views.
        Additionally, we need to measure network latency for parallel implementations as data is passed back and forth between the worker nodes. 

        \scriptsize
        \begin{table*}[h!]
        \centering
          \begin{tabular}{|c|c|c|c|c|}
            \hline
            {Benchmark Suite} & \multicolumn{4}{|c|}{Metrics} \\
            \hline
                                                      & ET & TEPS & LT & SLS\\
            \hline
             Graph500 [2010]\cite{Murphy2010}         &  X &      &    &     \\
             Social Benchmark [2013]\cite{Angles2013} &  X &      & X  &     \\
             Graphalytics [2015]\cite{Capota2015}     &  X &   X  &    &     \\
             GAP [2017]\cite{Beamer2017}              &  X &      &    &     \\
            \hline
          \end{tabular}
          \caption{Summary of Graph Benchmark Metrics.\\ ET = Execution Time, TEPS = Traversed Edges Per Second, LT = Load Time, SLS = Spatial Locality Score}
          \label{table:graphMetrics}
        \end{table*}
        \normalsize

        \subsection{System Architectures}\label{section:architecutres}
        %The HIVE project aims to achieve a 1000x improvement in graph processing by combining algorithmic and hardware approaches. 
        To evaluate the changes in the performance of differing hardware approaches, we need to compare implementations in sequential and parallel across CPU, GPU, and PiUMA system configurations. 
        CPU implementations are what most benchmarks have been developed for, almost exclusively in sequential configurations, less Graphalytics \cite{Capota2015}, which is designed for parallel evaluation. 
        GPUs are optimized for dense vector and matrix computation \cite{Dally2021} and are expected to perform relatively poorly on graph algorithms, at least relative to the significant gains seen in applications they are well suited for, like training Neural Networks. While the Graphalytics benchmark claims to support GPU evaluation, it lacks any discussion of results, and the veracity of the claims cannot be confirmed \cite{Capota2015}.
        PiMUA is a bespoke architecture from intel \cite{Aananthakrishnan2020}. 
        Prior work at ARLIS has evaluated the claimed performance improvements of PiUMA on the provided simulation and emulation platforms. 
        Only in the Summer of 2023 have we gained access to the single-PiUMA Software Development Variant (SDV) for running workloads and the Multi-PiUMA Optical Interconnect Assembly for measuring the latency of passing messages between multiple PiUMA chips in a distributed configuration. 
        While the Optical Interconnect Assembly is not functioning as designed due to a manufacturing flaw, we can use the measured latency to develop performance projections in conjunction with the SDV. 
        
        \scriptsize
        \begin{table*}[t]
        \centering
          \begin{tabular}{ |c|c|c|c|c|c|c|}
            \hline
            {Benchmark Suite} & \multicolumn{2}{|c|}{Implementation} & \multicolumn{3}{|c|}{Architecture}\\
            \hline
                                                      & Seq & Par & CPU & GPU & DSA \\
            \hline
             Graph500 [2010]\cite{Murphy2010}         & X   &     & X   &     &     \\
             Social Benchmark [2013]\cite{Angles2013} & X   &     & X   &     &     \\
             Graphalytics [2015]\cite{Capota2015}     & X   &  X  & X   &  ?  &     \\
             GAP [2017]\cite{Beamer2017}              & X   &     & X   &     &     \\
            \hline
          \end{tabular}
          \caption{Summary of Graph Benchmark Architectures.\\ Seq = Sequential, Par = Parallel/Distributed, CPU = Central Processing Unit, GPU = Graphics Processing Unit, DSA = Domain Specific Architecture}
          \label{table:graphArchitectures}
        \end{table*}
        \normalsize

    \subsection{Reference Algorithm Implementations}\label{section:referenceAlgorithms}
        \par{There are three core problem domains relevant to the Intelligence and Security community: Community Detection, Subgraph Matching, and Knowledge Graph Analytics. 
        We aim to develop reference implementations for dominant algorithms in these domains in both single-core and parallel processing configurations.
        As seen in table \ref{table:graphAlgorithms}, there is no standard set of reference algorithms for Graph Benchmarking because, depending on the domain of focus, the applicable algorithms vary. The lack of an established I\&S benchmark suite is driving us to collate our own reference implementations rather than use an existing benchmark. 
        %For Community Detection, we are adopting the Louvain Algorithm \cite{Blondel2008}. Reference implementations exist in C++ for sequential and distributed code \cite{Ghosh2018}. 
        
        To meet the goals of HIVE, we plan to focus benchmarking and evaluation efforts on three different graph processing algorithms, approaching the problem 'depth first':
    \newline
    
        \textbf{Subtask 1} Community Detection
        
        \textbf{Subtask 2} Subgraph Matching
        
        \textbf{Subtask 3} Knowledge Graphs


        %What is an example of the intelligence problem we're dealing with, e.g., we have a bunch of call data records
%We want to ask a question - e.g., who else is affiliated with some known bad influences
%We can represent CDRs as a graph and use a community detection algo to find that answer
%We use community detection algo of Louvain because of X, Y, and Z reasons (and cite the paper)
%The current limitation of the Louvian algo is X, which we will measure using Y to determine its impact
        \subsubsection{Community Detection}\label{section:communityDetection}
            In the intelligence community, community detection is widely used for identifying communities or groups within larger networks. 
            Say we are addressing a problem that involves analyzing a large set of call data records (CDRs) to identify potential associations with known bad influences. 
            To answer the question of who else may be affiliated with these influences, we can represent the CDRs as a graph and apply a community detection algorithm \cite{Truicua2018}. 
            One suitable algorithm is the Louvain Algorithm \cite{Blondel2008}, widely recognized for its effectiveness in identifying communities in complex networks. 
            The Louvain Algorithm is based on heuristic modularity optimization, which aims to enhance the density of links within communities and promote a cohesive community structure. 
            Its application in social network analysis and cybersecurity has proven valuable, enabling the detection of potential threats such as terrorist networks or malicious communities and botnets. The algorithm's is well-suited for large-scale and dynamic networks, such as the one represented by the CDRs. We consider the community detection deliverable complete once we have clearly outlined the Louvain Algorithm's expected behaviors, implemented a parallel and sequential version across CPU, GPU, and PiUMA architectures, and evaluated its performance with defined metrics across small, medium, and large datasets. 
            
        \subsubsection{Subgraph Matching}\label{section:subgraphMatching}
            In intelligence analysis, subgraph matching is crucial in uncovering hidden relationships within complex networks. 
            It is a graph-matching problem where the goal is to find occurrences of a smaller graph (subgraph) within a larger graph (target graph). 
            Identifying subgraphs that match specific patterns or structures could reveal relationships between entities, which can be valuable for intelligence analysis and decision-making. 
            For instance, we can consider investigating a social network to detect potential criminal networks involved in money laundering \cite{Soltani2016}. 
            By representing the social network as a graph, subgraph matching algorithms can be used to search for subgraphs that exhibit specific characteristics, such as a set of nodes representing individuals connected in a particular way, along with associated attributes like financial transactions, locations, and timestamps. 
            %Subgraph matching algorithms tailored to the pattern-matching problem can help identify clusters of nodes within the social network that exhibit similar patterns, revealing potential connections among individuals involved in illegal financial activities. 
            
            \par{Some suitable algorithms include graph isomorphism, network motif mining, and graph pattern matching. 
            The graph isomorphism algorithm \cite{Babai2016} checks for exact matches between the subgraph and target graph, which can be useful for identifying identical activity patterns. 
            The network motif mining algorithm \cite{Oliver2022}focuses on discovering frequently occurring subgraphs within the larger graph, which can reveal recurring structures or relationships among individuals. 
            The graph pattern matching algorithm \cite{Cheng2008} allows for the specification of complex patterns which can be used for specific patterns such as cycles or cliques in money laundering activities. 
            %There are many constraints to consider when deciding on an algorithm, as each approach is driven by the intelligence problem to be solved. 
            We are currently focused on selecting an appropriate reference algorithm for implementing subgraph matching. Subgraph matching accuracy is measured by evaluating two key aspects: existence and correctness. Existence focuses on determining whether a subgraph match is found or not in the target graph. Correctness assesses the accuracy of the identified matches using metrics such as precision, recall, and F1 score, considering the true positive, false positive, and false negative results. Labeled datasets with known ground truth are often used to compare the algorithm's output and compute these accuracy measures, while also considering factors like efficiency and scalability.} 

        \subsubsection{Knowledge Graphs}\label{section:knowledgeGraphs}
                
            Knowledge graph algorithms are designed to extract meaningful insights and uncover relationships within knowledge graphs, which represent information in a structured and interconnected manner.
            Common knowledge graph algorithms include entity linking and disambiguation, which connect entities within the graph to external knowledge bases.
            %Additionally, they include semantic similarity measures to determine the relatedness of entities based on their attributes.
            For example, consider the intelligence problem of counterterrorism. 
            A knowledge graph can represent entities under surveillance such as individuals, organizations, locations, and events, along with their relationships. 
            Entity linking and disambiguation algorithms can connect the entities within the knowledge graph to external databases such as watchlists or criminal records, aiding in identifying and tracking potential terrorist networks \cite{Xia2019}. 
            Semantic similarity measures can assess the relatedness of individuals based on their attributes, enabling the identification of potential associates or shared characteristics among suspects. Current semantic similarity methods include measuring the semantic network structure between concepts, focusing on path length and depth, while others focus on the Information Content (IC) of concepts. IC is a measure of specificity where higher values are associated with more specific concepts and lower values are associated with more general ideas \cite{Zhu2016}. To effectively benchmark knowledge graph analytics, further investigation and evaluation of these operations are necessary to establish standard metrics and performance evaluations across different architectures and technologies, however prior work in Knowledge Graph Analytics using a micro benchmarking approach to primitive operations of \textit{Selection, Adjacency, Reachability} and \textit{Summarization} \cite{Angles2013} offers a promising intial direction.
            
    \section{Project Timeline}\label{section:timeline}
    \osullikomment{Update post sprint with new numbers}
        \begin{center}
            \begin{tabular}{c|c|c}
                 Milestone              & Target Date   & Current Status\tablefootnote{Derived from Epic completion on Project Jira Board at \href{https://osullik.atlassian.net/jira/software/projects/HIVE/boards/1/}{https://osullik.atlassian.net/jira/software/projects/HIVE/boards/1/}}  \\
                 \hline
                 BFS                    & 6/9           & 36\% \\
                 Community Detection    & 6/23          & 59\% \\
                 Subgraph Matching      & 7/7           & 16\% \\
                 Knowledge Graph        & 7/21          & 0\% \\
                 Report and Brief       & 8/3           & 0\%
            \end{tabular}
            \label{table:timeline}
        \end{center}
        
    \section{Methodology}\label{section:methodology}
    We are approaching the project as an agile development problem, working 'depth-first' to implement benchmarks across each algorithm in turn.
    Our goal is to create a framework like the one depicted in figure \ref{fig:highLevelOverview}. 
    The remainder of the methodology section summarizes our progress so far and our plan to move forward across the Dataset Collection, Generating Metrics, Reference Implementations of the Algorithms and Establishing our experiment architecture.

    \begin{figure*}
            \centering
            \includegraphics[width=\columnwidth]{Papers/ARLIS_MPR_MN/Figures/high_level_overview.png}
            \caption{A high-level schematic diagram of the project framework, tasks, and deliverables}
            \label{fig:highLevelOverview}
        \end{figure*}

    
    \subsection{Dataset Collection and Generation}\label{section:datasetCreation}
        To conduct our initial testing, we developed a simple graph generator that takes three command-line arguments: the number of nodes in the graph, the number of edges, and a random seed and creates random (but reproducible) datasets. 
        For the actual benchmark datasets, where licensing and availability permits, we intend to use real-wold graph datasets to evaluate the execution speed, memory usage and power consumption of each implementation on each architecture.
        We will use synthetic graphs only where a suitable real-world graph is not available. 
        Given the classified nature of intelligence and security work, where we are not able to access a representative graph, if we are given statistical properties of a graph we sound use those to generate a 'similar' synthetic graph.
        
        %Future improvements to the creation tool involve curating custom datasets that reflect the structure and properties of real-world social networks, considering metrics such as cluster coefficient, assortativity, and distribution fit. 
        %Additionally, scale-free graphs are a valuable model for capturing characteristics of complex networks observed in domains such as social, biological, and technical networks \cite{Newman2003}. The pioneering work \cite{Barabasi1999} introduced the concept of scale-free networks, where the degree distribution follows a power-law distribution. Scale-free graphs adopt a preferential attachment mechanism \cite{Rak2020}where a few nodes, known as "hubs," have a significantly higher number of connections compared to the majority of nodes in the network, as new nodes tend to connect with already well-connected nodes. We acknowledge that relying solely on synthetic graph generation may only partially capture the complex nature of real-world networks. To address the synthetic data limitation, we also explore integrating real-world data sets, following the example of the GAP Benchmark \cite{Beamer2017}.
        
    \subsection{Metric Collection}\label{section:telemetry}
        Utilizing a standardized set of tools to evaluate algorithmic performance on multiple architectures is critical in attaining valid benchmarking results. 
        As mentioned in Section \ref{section:metrics}, we need to identify a standard to count the traversed edges per second, measure total execution time, measure spatial locality, measure temporal locality, and measure power consumption given each of these other views. 
        Additionally, we need to measure network latency for parallel implementations as data is passed back and forth between the workers. 
        We are experimenting with basic utilities such as 
        PERF \footnote{\href{https://perf.wiki.kernel.org/index.php/Main_Page}{https://perf.wiki.kernel.org/index.php/Main\_Page}}, 
        OPROFILE \footnote{\href{https://oprofile.sourceforge.io/about/}{https://oprofile.sourceforge.io/about/}}, and 
        GNU PROF \footnote{\href{http://web.archive.org/web/20141129061523/http://www.cs.utah.edu/dept/old/texinfo/as/gprof.html\#SEC2}{http://web.archive.org/web/20141129061523/http://www.cs.utah.edu/dept/old/texinfo/as/gprof.html\#SEC2}} as well as more robust application systems such as 
        HPC Toolkit \footnote{\href{http://hpctoolkit.org/}{http://hpctoolkit.org/}}, 
        VALGRIND \footnote{\href{https://valgrind.org/docs/manual/quick-start.html}{https://valgrind.org/docs/manual/quick-start.html}}, and 
        Intel VTUNE Profiler\footnote{\href{https://www.intel.com/content/www/us/en/developer/tools/oneapi/vtune-profiler.html\#gs.11c0nr}{https://www.intel.com/content/www/us/en/developer/tools/oneapi/vtune-profiler.html\#gs.11c0nr}}. 
        %The tools outlined above will enable us to capture the necessary benchmarking information, relying on an AWS EC2 instance to run the telemetry on an Ubuntu system. We are also considering adopting a CENTOS system to run on an EC2 Metal instance, as it appears to be a more suitable environment for running the PiUMA simulator.

    \subsection{Reference Algorithm Implementations}
        \par{We are identifying well reputed implementations of each algorithm to leverage for our benchmark. Some complexity exists because of the limited instruction set available to the PiUMA compiler. 
        We need to refactor each reference implementation to \textit{ANSI C} so that as closely as possible, we can use the same code can be compiled using $GCC$ for x86\_64 CPUs, with minor changes to compile in the CUDA environment with $NVCC$ for NVIDIA GPUs, and $PTK$ for the PiUMA}.
        By using the 'lowest common denominator' we are isolating the impact of confounding variables by ensuring that we are using the same algorithmic approaches to each of the implementations.}

    \subsection{Benchmark Platform}
        \par{We have established our evaluation environment for the CPU and GPU architectures on AWS. It is a $g3.0xlarge$ with 250Gb of Memory and a 32 Core Intel(R) Xeon(R) CPU E5-2686 v4 @ 2.30GHz. 
        It currently has 2 NVIDIA Tesla M6 GPUs and is running CUDA 12.2. 
        We have installed and configured all the relevant compilers and telemetry tooling discussed in section \ref{section:telemetry}.
        The evaluation on PiUMA will occur on a separate PiUMA Software Development Variant (SDV) for the sequential implementation, with paralell implementation likely to be deferred until manufacturing errors are rectified.}
       
    \section{Interfacing with Clients and Customers}\label{section:stakeholders}  
        \textbf{Agency:} DARPA\\  
        \textbf{POC:} Johnny Marsh \\
        \textbf{Date:} 7/12/23 \\
        \textbf{Purpose:} Communicate Progress on Tasks \\
        \textbf{High-level Takeaway(s):} N/A \\ 
        %\textit{[Repeat per engagement]} 

    \section{Next Steps}\label{section:nextSteps}
        The project has been moving slightly slower than anticipated regarding implementing and evaluating algorithms, yet making steady progress in setting up a robust experimental framework.
        An open-source implementation of the Sequential Louvain algorithm has been successfully implemented on a CPU architecture. 
        Telemetry capabilities have been integrated to measure time, memory, and power, providing valuable insights into algorithm performance. 
        The NVIDIA CUDA GPU architecture has also been configured, enabling us to compile and run our refrerence implementations on both CPU and GPU architectures. 
        %Finally, a dataset creation tool has been developed to produce graphs of various sizes. 
        The only major obstacle we are currently facing is not having access to the ARLIS PiUMA SDV. 
        Future anticipated challenges include identifying efficient implementations of subgraph matching and knowledge graph analytic primitives. 
        Looking ahead, the project will focus on implementing and evaluating subgraph matching and knowledge graph algorithms. 
        The telemetry capabilities will be expanded to include additional metrics and dimensions for a more comprehensive analysis. 
        The dataset curation process will continue evolving by incorporating more realistic and diverse data formats while considering scale-free graphs and real-world datasets. 
        By advancing these aspects, the project is on track to deliver benchmarking results of at least two out of three graph processing algorithms on both CPU and GPU architectures. 
        %Setting up the PiUMA framework remains a task in progress but is targeted for completion by the end of the month.
        %Overall progress is slow due to the necessity of setting up a comprehensive testing framework. Yet, it is on track to pick up pace as familiarity with tools/algorithms increases, ultimately contributing to HIVE's vision of improving the processing of streaming graphs at significantly higher speeds. 
        
    \section{References}
        \printbibliography[heading=none]

    \section{Acronyms}
        \small{
        \textbf{ARLIS} Applied Research Laboratory for Intelligence and Security.\\ 
        \textbf{BFS} Breadth-First Search \\
        \textbf{CPU} Central Processing Unit \\
        \textbf{DARPA} Defense Advanced Research Projects Agency \\ 
        \textbf{GPU} Graphics Processing Unit \\
        \textbf{HIVE} Hierarchical Identify Verify Exploit \\
        \textbf{KGA} Knowledge Graph Analytics \\
        \textbf{PIUMA} Programmable Integrated Unified Memory Architecture \\ 
        \textbf{SGM} Sub Graph Matching \\
        \textbf{TEPS} Traversed Edges Per Second \\
        \textbf{TEPS/W} Traversed Edges Per Second Per Watt
        }
    }
    \end{multicols}
    \newpage
\end{document}
