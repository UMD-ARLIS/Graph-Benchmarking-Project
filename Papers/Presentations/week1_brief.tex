% Original Code source: https://tex.stackexchange.com/questions/17700/quadrants-in-slides-using-lyx-and-beamer

\documentclass[usenames,dvipsnames]{beamer} %usenames and dvipsnames are extra parameters to allow for use of the extended colour pallet in xcolors

% % % % % % % % % % %
%Package Imports
% % % % % % % % % % %
\usepackage{caption}
\usepackage{xcolor}

% % % % % % % % % % %
% Custom Commands
% % % % % % % % % % %

%make a quadslide
\newcommand\FourQuad[4]{%
    \begin{minipage}[b][.35\textheight][t]{.47\textwidth}#1\end{minipage}\hfill%
    \begin{minipage}[b][.35\textheight][t]{.50\textwidth}#2\end{minipage}\\[0.5em]
    \begin{minipage}[b][.35\textheight][t]{.47\textwidth}#3\end{minipage}\hfill
    \begin{minipage}[b][.35\textheight][t]{.50\textwidth}#4\end{minipage}\\
    \includegraphics[scale=0.05]{Papers/Presentations/Figures/arlis.png} %not best way, need to put in nicer
}

% % % % % % % % % % %
% Document Content
% % % % % % % % % % %

\begin{document}

    \begin{frame}
        \frametitle{\textcolor{Mahogany}{Week 1 - Team MN: Hierarchical Identify Verify Exploit}}
        \FourQuad%
        %Top Left
        {
            \tiny{
                \textbf{High Level Project Goal}  
                \begin{itemize}
                    \item Benchmark PiUMA, a new architecture for high-performance graph processing utilizing against the provided projections 
                    \item Emphasis on comparative benchmarking of PiUMAs graph processing capabilities in community detection, subgraph matching and knowledge graph analytics against CPU and GPU architectures.
                \end{itemize} 
            }
        }%
        %Top Right
        {
            \tiny{
                \textbf{Project Approach - how will you accomplish the goal?} 
                \begin{itemize}
                    \item Implementing sequential and parallel applications on the CPU, GPU, and PiUMA, to leverage their respective computational capabilities.
                    \item Analyzing an appropriate variety datasets, ensuring comprehensive coverage and representative data for accurate evaluation and testing of the systems' capabilities.
                    \item Evaluate and measure the performance of the systems using benchmarking methodologies, collecting and analyzing performance metrics to assess the effectiveness.
                \end{itemize}
            }
        }%
        %Bottom Left
        {
            \tiny{
                \textbf{Initial Steps - what are you doing next week?}
                \begin{itemize}
                    \item Implement both sequential and parallel BFS algorithms across several computational platforms, encompassing the CPU, GPU, and the PiUMA architecture. 
                    \item Create datasets across varying scales, encompassing a variety of sizes, utilizing the Dataset creation tool developed by our team. 
                    \item Secure remote access to the PiUMA Emulation Environment, and subsequently establish the PiUMA SDV
                \end{itemize}   
            }
        }%
        %Bottom Right
        {
            \tiny{
                \textbf{Significance if project is successful}
                \begin{itemize}
                    \item Revolutionize graph analytics by enabling real-time processing of streaming graphs 1000X faster and with significantly lower power consumption, empowering timely decision-making in various domains.  
                    \item Inspire further research and development in graph analytics, fostering innovation and driving progress in related fields while promoting sustainability through energy-efficient graph processing solutions.
                \end{itemize} 
            }
        }
    \end{frame}
    
\end{document}