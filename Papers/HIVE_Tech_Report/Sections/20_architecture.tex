\section{Benchmark Architecture}\label{section:architecture}
Our benchmark consists of \textit{Datasets} specific to I\&S problem domains, \textit{Reference Implementations} of graph processing algorithms useful to the I\&S community and a suite of tools to collect the \textit{Telemetry} required to determine whether a given hardware platform is better suited for a specific graph processing task. 

\subsection{Design Philosophy}
The AGHAB is driven by empirical methods in computer science~\cite{Hooker1994, Hooker1995, Mcgeoch1996}.
Broadly, rather than developing a `leaderboard' to identify which particular hardware accelerator is `best', we aim to develop a platform to facilitate research exploring how the interactions of the problem datasets, reference implementations and hardware accelerators relate to the \textit{effectiveness} and \textit{efficency} of a problem. 
Here, \textit{effectiveness} is the fitness for solving a problem and measures the correctness and quality of the solution produced by the reference implementation. 
The \textit{efficency} measures the throughput, resource consumption, and execution speed.
An optimal solution is efficient and effective, but most approaches accept degradation in one for improvement in the other. 
The AGHAB is designed to be transparent, reproducible and portable.

\subsection{Operational View 1}
Figure~\ref{figure:ov1} shows the high-level use case for the AGHAB. 
An evaluation using AGHAB begins with downloading the benchmark from the ARLIS GitHub. 
They unpack the benchmark datasets, reference implementations and telemetry suite, and define experiments to run.
An \textit{experiment} is a single run combining a dataset, reference implementation and hardware platform.
The experiment specification drives the experiment framework, controlling the compilation, dataset generation, instrumentation, execution, telemetry and reporting of the benchmark system.
Related experiments are collected into an \textit{experiment suite}.
An evaluator can define their own compiler and other external dependencies in the experiment specification to facilitate deploying the benchmark on their hardware. 
Once the experiment suite is complete, the evaluator receives a report created from the collected telemetry, and the system adds the results to the records. 
\begin{figure*}
    \centering
    \includesvg[width=0.8\textwidth]{ov1}
    \label{figure:ov1}
    \caption{A user of the AGHAB provides their hardware, optionally any required compiler and a specification of the experiments to run.}
\end{figure*}
\begin{figure*}
    \centering
    \includesvg[width=0.8\textwidth]{ov2}
    \label{figure:ov2}
    \caption{The Experiment Framework reads the experiment specification and collects the specified datasets, the reference implementations and creates an experiment to task the baseline CPU, GPU and Graph Hardware Accelerators (GHA). It controls the data collection and report generation.}
\end{figure*}

\subsection{Operational View 2}
The three components of the AGHAB shown in figure~\ref{figure:ov2} control variance in the datasets, implementation and measurement so that the focus of the variable under exam remains the hardware platform. 
The experiment defines the dataset(s) to be tested against.
The datasets are usually pre-generated but may be dynamically generated if required. 
The experiment defines the reference implementation to use, and facilitates compilation where required.
If passes the dataset to the reference implementation, and evaluates the output against the ground-truth to generate the measures of \textit{effectiveness}.
The experiment framework initializes the telemetry suite, and controls the collection, fusion and analysis of telemetry. 
The results of the analysis are passed to the reporting module, which generates reports for the evaluator, and posts results to the system for tracking. 



