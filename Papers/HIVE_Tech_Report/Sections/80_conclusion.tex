\section{Conclusion}\label{section:conclusion}
The ARLIS Graph Hardware Accelerator Benchmark (AGHAB) provides the framework to evaluate the impact of incorporating graph hardware accelerators into Intelligence and Security (I\&S) workflows. 
Specifically, it serves as a mechanism to decide which, if any, accelerator is suitable to improve the speed, throughput or resource consumption of a graph processing task.
It provides a platform and a framework for further empirical study of graph-based analytics in time and resource-constrained environments. 
The AGHAB is extensible and designed to serve as the abstraction layer between the experimenter and the hardware. 
The experimenter need only provide the hardware, relevant compilers and experiment definitions to link a new graph hardware accelerator into AGHAB.
Then, it can leverage the datasets, analytics, reference implementations and baselines built into the experimental framework to run experiments, collect data and generate reports. 
AGHAB is in its protoform at present, but continuing work will create a platform to evaluate the rise of Graph-Based Artificial Intelligence and the impact of hardware acceleration on Graph-Based AI into the future. 