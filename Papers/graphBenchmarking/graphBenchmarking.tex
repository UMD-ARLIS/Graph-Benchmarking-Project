%%
%% This is file `sample-authordraft.tex',
%% generated with the docstrip utility.
%%
%% The original source files were:
%%
%% samples.dtx  (with options: `authordraft')
%% 
%% IMPORTANT NOTICE:
%% 
%% For the copyright see the source file.
%% 
%% Any modified versions of this file must be renamed
%% with new filenames distinct from sample-authordraft.tex.
%% 
%% For distribution of the original source see the terms
%% for copying and modification in the file samples.dtx.
%% 
%% This generated file may be distributed as long as the
%% original source files, as listed above, are part of the
%% same distribution. (The sources need not necessarily be
%% in the same archive or directory.)
%%
%% Commands for TeXCount
%TC:macro \cite [option:text,text]
%TC:macro \citep [option:text,text]
%TC:macro \citet [option:text,text]
%TC:envir table 0 1
%TC:envir table* 0 1
%TC:envir tabular [ignore] word
%TC:envir displaymath 0 word
%TC:envir math 0 word
%TC:envir comment 0 0
%%
%%
%% The first command in your LaTeX source must be the \documentclass command.
\documentclass[sigconf,authordraft]{acmart}
%% NOTE that a single column version may required for 
%% submission and peer review. This can be done by changing
%% the \doucmentclass[...]{acmart} in this template to 
%% \documentclass[manuscript,screen]{acmart}
%% 
%% To ensure 100% compatibility, please check the white list of
%% approved LaTeX packages to be used with the Master Article Template at
%% https://www.acm.org/publications/taps/whitelist-of-latex-packages 
%% before creating your document. The white list page provides 
%% information on how to submit additional LaTeX packages for 
%% review and adoption.
%% Fonts used in the template cannot be substituted; margin 
%% adjustments are not allowed.

%%
%% \BibTeX command to typeset BibTeX logo in the docs
\AtBeginDocument{%
  \providecommand\BibTeX{{%
    \normalfont B\kern-0.5em{\scshape i\kern-0.25em b}\kern-0.8em\TeX}}}

%% Rights management information.  This information is sent to you
%% when you complete the rights form.  These commands have SAMPLE
%% values in them; it is your responsibility as an author to replace
%% the commands and values with those provided to you when you
%% complete the rights form.
\setcopyright{acmcopyright}
\copyrightyear{2018}
\acmYear{2018}
\acmDOI{XXXXXXX.XXXXXXX}

%% These commands are for a PROCEEDINGS abstract or paper.
\acmConference[Conference acronym 'XX]{Make sure to enter the correct
  conference title from your rights confirmation emai}{June 03--05,
  2018}{Woodstock, NY}
%
%  Uncomment \acmBooktitle if th title of the proceedings is different
%  from ``Proceedings of ...''!
%
%\acmBooktitle{Woodstock '18: ACM Symposium on Neural Gaze Detection,
%  June 03--05, 2018, Woodstock, NY} 
\acmPrice{15.00}
\acmISBN{978-1-4503-XXXX-X/18/06}


%%
%% Submission ID.
%% Use this when submitting an article to a sponsored event. You'll
%% receive a unique submission ID from the organizers
%% of the event, and this ID should be used as the parameter to this command.
%%\acmSubmissionID{123-A56-BU3}

%%
%% For managing citations, it is recommended to use bibliography
%% files in BibTeX format.
%%
%% You can then either use BibTeX with the ACM-Reference-Format style,
%% or BibLaTeX with the acmnumeric or acmauthoryear sytles, that include
%% support for advanced citation of software artefact from the
%% biblatex-software package, also separately available on CTAN.
%%
%% Look at the sample-*-biblatex.tex files for templates showcasing
%% the biblatex styles.
%%

%%
%% For managing citations, it is recommended to use bibliography
%% files in BibTeX format.
%%
%% You can then either use BibTeX with the ACM-Reference-Format style,
%% or BibLaTeX with the acmnumeric or acmauthoryear sytles, that include
%% support for advanced citation of software artefact from the
%% biblatex-software package, also separately available on CTAN.
%%
%% Look at the sample-*-biblatex.tex files for templates showcasing
%% the biblatex styles.
%%

%%
%% The majority of ACM publications use numbered citations and
%% references.  The command \citestyle{authoryear} switches to the
%% "author year" style.
%%
%% If you are preparing content for an event
%% sponsored by ACM SIGGRAPH, you must use the "author year" style of
%% citations and references.
%% Uncommenting
%% the next command will enable that style.
%%\citestyle{acmauthoryear}

%%
%% end of the preamble, start of the body of the document source.
\begin{document}

%%
%% The "title" command has an optional parameter,
%% allowing the author to define a "short title" to be used in page headers.
\title{The Name of the Title is Hope}

%%
%% The "author" command and its associated commands are used to define
%% the authors and their affiliations.
%% Of note is the shared affiliation of the first two authors, and the
%% "authornote" and "authornotemark" commands
%% used to denote shared contribution to the research.
\author{Kent O'Sullivan}
%\authornote{Both authors contributed equally to this research.}
\email{osullik@umd.edu}
%\orcid{1234-5678-9012}
%\author{G.K.M. Tobin}
%\authornotemark[1]
%\email{webmaster@marysville-ohio.com}
\affiliation{%
  \institution{University of Maryland}
  %\streetaddress{P.O. Box 1212}
  %\city{Dublin}
  %\state{Ohio}
  \country{USA}
  %\postcode{43017-6221}
}

\author{Morein Ibrahim}
\email{morein04@terpmail.umd.edu}
\affiliation{%
  \institution{The University of Maryland}
  %\streetaddress{1 Th{\o}rv{\"a}ld Circle}
  %\city{Hekla}
  \country{USA}}


\author{Nandini Ramachandran}
\email{nandinir@terpmail.umd.edu}
\affiliation{%
  \institution{University of Maryland}
  %\city{Rocquencourt}
  \country{USA}
}

\author{Cliston Cole}
\email{cliston.cole@morgan.edu}
\affiliation{%
 \institution{Morgan State University}
 %\streetaddress{Rono-Hills}
 %\city{Doimukh}
 %\state{Arunachal Pradesh}
 \country{University of Maryland}}

\author{William Regli}
\email{regli@umd.edu}
\affiliation{%
  \institution{University of Maryland}
  %\streetaddress{30 Shuangqing Rd}
  %\city{Haidian Qu}
  %\state{Beijing Shi}
  \country{USA}}

%%
%% By default, the full list of authors will be used in the page
%% headers. Often, this list is too long, and will overlap
%% other information printed in the page headers. This command allows
%% the author to define a more concise list
%% of authors' names for this purpose.
\renewcommand{\shortauthors}{Trovato and Tobin, et al.}

%%
%% The abstract is a short summary of the work to be presented in the
%% article.
\begin{abstract}
\end{abstract}

%%
%% The code below is generated by the tool at http://dl.acm.org/ccs.cfm.
%% Please copy and paste the code instead of the example below.
%%
\begin{CCSXML}
  <ccs2012>
     <concept>
         <concept_id>10002944.10011123.10011130</concept_id>
         <concept_desc>General and reference~Evaluation</concept_desc>
         <concept_significance>500</concept_significance>
         </concept>
     <concept>
         <concept_id>10002944.10011123.10010916</concept_id>
         <concept_desc>General and reference~Measurement</concept_desc>
         <concept_significance>500</concept_significance>
         </concept>
     <concept>
         <concept_id>10002944.10011123.10011674</concept_id>
         <concept_desc>General and reference~Performance</concept_desc>
         <concept_significance>500</concept_significance>
         </concept>
     <concept>
         <concept_id>10010520.10010521.10010542.10011714</concept_id>
         <concept_desc>Computer systems organization~Special purpose systems</concept_desc>
         <concept_significance>300</concept_significance>
         </concept>
   </ccs2012>
\end{CCSXML}
  
\ccsdesc[500]{General and reference~Evaluation}
\ccsdesc[500]{General and reference~Measurement}
\ccsdesc[500]{General and reference~Performance}
\ccsdesc[300]{Computer systems organization~Special purpose systems}

%%
%% Keywords. The author(s) should pick words that accurately describe
%% the work being presented. Separate the keywords with commas.
\keywords{}

%% A "teaser" image appears between the author and affiliation
%% information and the body of the document, and typically spans the
%% page.
%\begin{teaserfigure}
%  \includegraphics[width=\textwidth]{sampleteaser}
%  \caption{Seattle Mariners at Spring Training, 2010.}
%  \Description{Enjoying the baseball game from the third-base
%  seats. Ichiro Suzuki preparing to bat.}
%  \label{fig:teaser}
%\end{teaserfigure}

\received{20 February 2007}
\received[revised]{12 March 2009}
\received[accepted]{5 June 2009}

%%
%% This command processes the author and affiliation and title
%% information and builds the first part of the formatted document.
\maketitle

\section{Introduction}


\section{Related Work}

\subsection{System Benchmarking}

\par{A 2005 Study by Weinberg et. al. examines the measurement of spatial and temporal locality in high performance computing. 
They define spatial locality as memeory addresses which are located close to each other, and temporal locality as the same memory addresses being accessed repeatedly over time. 
They conduct their measurements by instrumenting their code using the MEMSIM platform, and use it to generate a spatial and temporal locality index with parameters L and K. 
L is a measurement of the stride size, and K is a measure of the randomness of access. 
They present these metrics despite earlier warnings in their own work about the reductive effects of generating an index to represent locality. 
Their analysis suggests that a lack of locality degrades the execution time of applications and so it is interesting that neither Graph500 nor the newer graph benchmark suites account for the loading or structuring of graphs in memory in their measurement. 
Chatacterising the shape of graphs in memory appears to be an under-explored dimension of the benchmarking space \cite{Weinberg2005}.}

\par{A 2010 description of how Google implements system profiling across its data warehouses by Ren et. al. focuses on continuous monitoring rather than benchmarking.  
Their experience argues that sampling binaries during execution rather than fully instrumenting at compile time them is a superior approach that reduces memory usage and execution time. 
They sample events, which can include clock cycles, L1 and L2 cache misses and branch mispredictions. 
Their work provides a precedent for using profiling to compare different hadware implementations of the same application, supporting our evaluation of Graph Algorithms across CPU, GPU and PiUMA for the HIVE project. 
There appears to be a gap in defining what a standard 'profile' is for a graph algorithm. 
Detemining what a standard 'profile', and further determining a method to effectively vizualize memory accesses for graph applications by time and locality will be a prosperous avenue of further research \cite{Ren2010}.}

\par{In 2009 Adhianto et. al. from Rice University released the initial build of the High Performance Computing Toolkit (HPCToolkit). 
The toolkit aims to profile the performance of code on high performance systems without needing to instrument the code, reducing the overheads imposed on the system under test. 
Their approach and toolkit exemplifies the measurement of perforance without instrumentation, demonstrating the feasibility of profiling code without intrinsicly degrading its performance \cite{Adhianto2010}.}

\subsection{Graph Benchmarking}


%%
%% The acknowledgments section is defined using the "acks" environment
%% (and NOT an unnumbered section). This ensures the proper
%% identification of the section in the article metadata, and the
%% consistent spelling of the heading.
\begin{acks}
To Robert, for the bagels and explaining CMYK and color spaces.
\end{acks}

%%
%% The next two lines define the bibliography style to be used, and
%% the bibliography file.
\bibliographystyle{ACM-Reference-Format}
\bibliography{../bibliography/summer2023.bib}

%%
%% If your work has an appendix, this is the place to put it.
%\appendix


\end{document}
\endinput
%%
%% End of file `sample-authordraft.tex'.
