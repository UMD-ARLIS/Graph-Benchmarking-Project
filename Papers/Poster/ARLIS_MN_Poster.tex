%%%%%%%%%%%%%%%%%%%%%%%%%%%%%%%%%%%%%%%%%
% baposter Landscape Poster
% LaTeX Template
% Version 1.0 (11/06/13)
%
% baposter Class Created by:
% Brian Amberg (baposter@brian-amberg.de)
%
% This template has been downloaded from:
% http://www.LaTeXTemplates.com
%
% License:
% CC BY-NC-SA 3.0 (http://creativecommons.org/licenses/by-nc-sa/3.0/)
%
%%%%%%%%%%%%%%%%%%%%%%%%%%%%%%%%%%%%%%%%%

%----------------------------------------------------------------------------------------
%	PACKAGES AND OTHER DOCUMENT CONFIGURATIONS
%----------------------------------------------------------------------------------------

\documentclass[landscape,a0paper,fontscale=0.285]{baposter} % Adjust the font scale/size here


\usepackage{hyperref}

\usepackage{graphicx} % Required for including images
\usepackage{svg}
\graphicspath{{figures/}} % Directory in which figures are stored

\usepackage{amsmath} % For typesetting math
\usepackage{amssymb} % Adds new symbols to be used in math mode

\usepackage{booktabs} % Top and bottom rules for tables
\usepackage{enumitem} % Used to reduce itemize/enumerate spacing
\usepackage{palatino} % Use the Palatino font
\usepackage[font=small,labelfont=bf]{caption} % Required for specifying captions to tables and figures

\usepackage{multicol} % Required for multiple columns
\setlength{\columnsep}{1.5em} % Slightly increase the space between columns
\setlength{\columnseprule}{0mm} % No horizontal rule between columns

\usepackage{tikz} % Required for flow chart
\usetikzlibrary{shapes,arrows} % Tikz libraries required for the flow chart in the template

\newcommand{\compresslist}{ % Define a command to reduce spacing within itemize/enumerate environments, this is used right after \begin{itemize} or \begin{enumerate}
\setlength{\itemsep}{1pt}
\setlength{\parskip}{0pt}
\setlength{\parsep}{0pt}
}

\definecolor{umd_yellow}{RGB}{255,210,1}
\definecolor{umd_red}{RGB}{227,25,51} % Defines the color used for content box headers

\begin{document}

\begin{poster}
{
headerborder=closed, % Adds a border around the header of content boxes
colspacing=1em, % Column spacing
bgColorOne=white, % Background color for the gradient on the left side of the poster
bgColorTwo=white, % Background color for the gradient on the right side of the poster
borderColor=black, % Border color
headerColorOne=umd_red, % Background color for the header in the content boxes (left side)
headerColorTwo=umd_red, % Background color for the header in the content boxes (right side)
headerFontColor=white, % Text color for the header text in the content boxes
boxColorOne=white, % Background color of the content boxes
textborder=roundedleft, % Format of the border around content boxes, can be: none, bars, coils, triangles, rectangle, rounded, roundedsmall, roundedright or faded
eyecatcher=true, % Set to false for ignoring the left logo in the title and move the title left
headerheight=0.1\textheight, % Height of the header
headershape=roundedright, % Specify the rounded corner in the content box headers, can be: rectangle, small-rounded, roundedright, roundedleft or rounded
headerfont=\Large\bf\textsc, % Large, bold and sans serif font in the headers of content boxes
%textfont={\setlength{\parindent}{1.5em}}, % Uncomment for paragraph indentation
linewidth=2pt % Width of the border lines around content boxes
}
%----------------------------------------------------------------------------------------
%	TITLE SECTION 
%----------------------------------------------------------------------------------------
%
{\includegraphics[height=4em]{arlis.png}} % First university/lab logo on the left
{\bf\textsc{MINNESOTA: Benchmarking Graph Processing}\vspace{0.2em}} % Poster title
{\textsc{Morein Ibrahim, Nandini Ramachandran\\ \hspace{12pt} Kent O'Sullivan, Cliston Cole, \& William Regli}} % Author names and institution
{\includegraphics[height=6em]{qrcode.png}}

%----------------------------------------------------------------------------------------
%	OBJECTIVES
%----------------------------------------------------------------------------------------

\headerbox{1. Objectives}{name=objectives,column=0,row=0}{

Develop a benchmarking suite to measure the progress of Intel's novel graph processing architecture: the Programmable Integrated Unified Memory Architecture (PiUMA): 

\begin{enumerate}
    \item Collect datasets that represent relevant intelligence and security problems. 
    \item Source or develop suitable cross-architecture reference implementations of graph analysis algorithms useful to the Intelligence and Security (I\&S) community.
    \item Construct an experimental framework to report performance (execution time, memory usage, and power consumption).
    \item Benchmark PiUMA; a new architecture for high-performance graph processing against the provided projections \cite{Aananthakrishnan2020}.
\end{enumerate}

\vspace{0.3em} % When there are two boxes, some whitespace may need to be added if the one on the right has more content
}

%----------------------------------------------------------------------------------------
%	INTRODUCTION
%----------------------------------------------------------------------------------------

\headerbox{2. Why graph processing?}{name=problem,column=1,row=0}{
Graph processing is used in the I\&S domain to address complex interdependencies in real-world systems, analyze data efficiently, and detect anomalies and patterns for security purposes.\\

To understand the benefit of PiUMA in graph analysis, we compared graph processing performance on real-world datasets in areas of interest to the Intelligence and Security Community.

}

\headerbox{3. Reference Algorithms}{name=algorithms,column=1, row=0, bottomaligned=objectives, below=problem}{
\textbf{Louvain.} Algorithm to detect communities in graphs by heuristic optimization of modularity \cite{Blondel2008}. Tested on 26GB of tweets to find communities.

\textbf{VF3.} Algorithm to find isomorphic sub-graph patterns in larger graphs \cite{Carletti2017}. Tested on a Financial dataset to detect suspicious transaction patterns. 
%Many intelligence and security problems are fundamentally social problems that can be modeled as a collection of social interactions between humans using a graph data structure. 
%The HIVE program seeks to develop an advanced graph processor capable of efficiently processing streaming graphs 1000x faster while reducing power consumption. 
%The motivating problem is that when graphs get very large, they become very slow to search, and many problems quickly become intractable. 
%HIVE expects that specialized hardware optimized for Graph Processing may make some of these ’intractable’ problems solvable, and expand the toolkit available to the I\&S community. 
%Towards determining whether the PiUMA system is a step towards the HIVE goal, we consider our research in the context of prior work in graph workload arking.

}

%----------------------------------------------------------------------------------------
%	BENCHMARK FRAMEWORK
%----------------------------------------------------------------------------------------

\headerbox{4. Benchmark Framework}{name=method,column=0, span=2, below=objectives,above=bottom}{ % This block's bottom aligns with the bottom of the conclusion block

\vspace{1em}
\begin{center}
\includesvg[width=0.80\linewidth]{system_diagram.svg}
\captionof{figure}{Our benchmark framework as prototyped on an AWS P5.8x large instance. It includes reference datasets, sequential and parallel implementations of the Louvain and VF3 algorithms and a suite of measurement tools to create comparison reports between CPU, GPU and PiUMA architectures for Graph Processing.} 

\end{center}

}

%----------------------------------------------------------------------------------------
%	RESULTS 1
%----------------------------------------------------------------------------------------

\headerbox{5. Summary of Results}{name=graphs,column=2,span=2,row=0}{

\begin{multicols}{2}
\vspace{1em}
\begin{center}

\includegraphics[width=0.55\linewidth]{placeholder.jpg}
\captionof{figure}{Results to be pasted on top of poster.}

\includegraphics[width=0.55\linewidth]{placeholder.jpg}
\captionof{figure}{Results to be pasted on top of poster.}





\end{center}
\end{multicols}

%------------------------------------------------

\begin{multicols}{2}
\vspace{1em}



\begin{center}
    
\includegraphics[width=0.55\linewidth]{placeholder.jpg}
\captionof{figure}{Results to be pasted on top of poster.}

\includegraphics[width=0.55\linewidth]{placeholder.jpg}
\captionof{figure}{Results to be pasted on top of poster.}

\end{center}

%\textbf{What causes False Positives?} Keeping Many Secrets. The system becomes 'paranoid' as the probability of returning an incorrect answer with a high cosine similarity (e.g. numeric answers) increases. \\
%\textbf{What causes False Negatives?} Information leakage results from a lack of detail in the secret context and also from 'interrogation' with large volumes of probing questions. 

\end{multicols}

\begin{multicols}{2}
\vspace{1em}



\begin{center}
    
\includegraphics[width=0.55\linewidth]{placeholder.jpg}
\captionof{figure}{Results to be pasted on top of poster.}

\includegraphics[width=0.55\linewidth]{placeholder.jpg}
\captionof{figure}{Results to be pasted on top of poster.}

\end{center}

%\textbf{What causes False Positives?} Keeping Many Secrets. The system becomes 'paranoid' as the probability of returning an incorrect answer with a high cosine similarity (e.g. numeric answers) increases. \\
%\textbf{What causes False Negatives?} Information leakage results from a lack of detail in the secret context and also from 'interrogation' with large volumes of probing questions. 

\end{multicols}
}

%----------------------------------------------------------------------------------------
%	REFERENCES
%----------------------------------------------------------------------------------------

\headerbox{8. References}{name=references,column=2,span=2, above=bottom}{

\renewcommand{\section}[2]{\vskip 0.05em} % Get rid of the default "References" section title
%\nocite{*} % Insert publications even if they are not cited in the poster
\tiny{ % Reduce the font size in this block
\bibliographystyle{unsrt}

\begin{multicols}{2}
   \bibliography{Papers/bibliography/summer2023} % Use sample.bib as the bibliography file 
\end{multicols}

}}

%----------------------------------------------------------------------------------------
%	FUTURE RESEARCH
%----------------------------------------------------------------------------------------

%\headerbox{7. Future Research}{name=futureresearch,column=1,span=2,aligned=references,above=bottom}{ % This block is as tall as the references block

%\begin{multicols}{2}

%\end{multicols}
%}

%----------------------------------------------------------------------------------------
%	CONTACT INFORMATION
%----------------------------------------------------------------------------------------

%\headerbox{9. Contact Information}{name=contact,column=3,aligned=references,above=bottom}{ % This block is as tall as the references block

%\begin{description}\compresslist
%\item[Github]\href{https://github.com/osullik/WhoIsYourDaddyAndWhatDoesHeDo}{Link to project Git}
%\item[Email] \{sakshumk, bmoskow1, osullik, nrolling, rjvanv\}@umd.edu
%\end{description}
%}

%----------------------------------------------------------------------------------------
%	CONCLUSION
%----------------------------------------------------------------------------------------

\headerbox{6. Future Work}{name=future_work,column=2,row=0,below=graphs, above=references}{

We have focused on developing the \textit{framework} for benchmarking. Immediate next steps include: 

\begin{itemize}\compresslist
    \item {Move benchmark framework from the cloud to a physical system for 'fair' benchmarking that can leverage the analytic tools that are limited in the cloud, like \textit{perf}.}
    \item{Refactor the CPU, GPU and PiUMA reference algorithm implementations to a codebase to improve fairness of comparison}
    \item{Run parallel benchmark on PiUMA Optical Interconnect.}
\end{itemize}



% \begin{multicols}{2}

% \tikzstyle{decision} = [diamond, draw, fill=blue!20, text width=4.5em, text badly centered, node distance=2cm, inner sep=0pt]
% \tikzstyle{block} = [rectangle, draw, fill=blue!20, text width=5em, text centered, rounded corners, minimum height=4em]
% \tikzstyle{line} = [draw, -latex']
% \tikzstyle{cloud} = [draw, ellipse, fill=red!20, node distance=3cm, minimum height=2em]

% \begin{tikzpicture}[node distance = 2cm, auto]
% \node [block] (init) {Initialize Model};
% \node [cloud, left of=init] (Start) {Start};
% \node [cloud, right of=init] (Start2) {Start Two};
% \node [block, below of=init] (init2) {Initialize Two};
% \node [decision, below of=init2] (End) {End};
% \path [line] (init) -- (init2);
% \path [line] (init2) -- (End);
% \path [line, dashed] (Start) -- (init);
% \path [line, dashed] (Start2) -- (init);
% \path [line, dashed] (Start2) |- (init2);
% \end{tikzpicture}

%------------------------------------------------



% \end{multicols}
}

\headerbox{7. Conclusion}{name=conclusion,column=3,row=0,below=graphs, above=references}{
Our most significant contributions are:

\begin{itemize}\compresslist
\item Identification and implementation of graph algorithms and datasets relevant to I \& S problems.
\item Development of a framework to compare the execution time and memory usage of those algorithms on those datasets on CPU, GPU and PiUMA architectures.
\end{itemize}

We gratefully acknowledge the supprt of our Mentors, ARLIS and the DARPA HIVE program. 

%The main limitations to this method are:
%\begin{itemize}\compresslist
%\item Possible hidden complications with open-sourced algorithms.
%\item Inaccuracies in data results.
%\item Time constraints with data collection.
%\end{itemize}
}



\end{poster}

\end{document}